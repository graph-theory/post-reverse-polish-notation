\documentclass[addpoints,12pt]{exam}


\makeatletter
\expandafter\providecommand\expandafter*\csname ver@framed.sty\endcsname
{2003/07/21 v0.8a Simulated by exam}
\makeatother

\usepackage{xcolor}
\usepackage{minted}
\usepackage[utf8]{inputenc}
\usepackage{tikz}
\usepackage{caption}
\usepackage{gensymb}
\usepackage{lmodern}
\usepackage{multirow}
\usepackage{booktabs}
\usepackage{array}
\usepackage{adjustbox}
\usepackage{upquote}
\usepackage{amsmath}
\usepackage[hidelinks]{hyperref}
\usetikzlibrary{mindmap,shadows, shapes, arrows, positioning}

\tikzstyle{rect} = [rectangle, fill=ProcessBlue, text width=4.5em, text centered, minimum height=4em, rounded corners]
\tikzstyle{line} = [draw, ->, very thick]
\tikzstyle{oval} = [ellipse, fill=SeaGreen, text width=5em, text centered]

\newcolumntype{x}[1]{>{\centering\arraybackslash\hspace{0pt}}p{#1}}

\renewcommand{\refname}{\selectfont\normalsize References} 
\pagestyle{headandfoot}

\header{\textbf{Problem Sheet: MU}}{}{Theory of Algorithms}
\footer{}{Page \thepage\ of \numpages}{}
\marksnotpoints
\pointsinrightmargin

\begin{coverpages}
\end{coverpages}

\begin{document}
%!TEX root = problems.tex

%\printanswers

\noindent
Reverse Polish notation is a method for writing expressions involving operators and operands.
When the number of operands that each operator takes is fixed, reverse Polish notation does not require any brackets or precedence of operators in order to unambiguously represent an expression.

\subsubsection*{Example}
Let’s start with an example.
Consider the following bracketed expression.

$$ ((5 + 4) \times 9) \div (6 - 3) $$

This expression is written in what we call infix notation -- the (binary) operators are written inbetween their (two) operands.
In reverse Polish notation, this expression is written as:

$$ 3 \quad 6 \quad - \quad 9 \quad 5 \quad 4 \quad + \quad \times \quad \div $$

Aside from the benefit of not needing brackets, there’s another reason that reverse Polish notation is popular in computing.
There is a simple algorithm for evaluating expressions written in reverse Polish notation, and it only requires a single pass through the expression.

\bibliographystyle{plain}
\bibliography{bibliography}
\end{document}
